\subsection{Модификация библиотеки}

\subsubsection{Состояние}

Для сохранения рабочей информации о ходе анализа в \angstrom{} была добавлена возможность сохранения и передачи состояния.
Модуль \verb|Angstrom| был переделан в функтор с параметром \lstinline{(State : sig type t end)},
тип парсера был изменён
\begin{lstlisting}
    type 'a with_state = input -> State.t -> 'a
\end{lstlisting}
и был добавлен набор парсеров для взаимодействия с состоянием:
\begin{lstlisting}
    val get_state : State.t t
    val set_state : State.t -> unit t
    val map_state : (State.t -> State.t) -> unit t
\end{lstlisting}

\subsubsection{Позиция}

В существующем анализаторе \rescript{} для обозначения позиции используются номер строки и смещение от её начала в байтах.
Причём номер строки считается по количеству символов \textsc{LF} и последовательностей \textsc{CRLF}.
Для получения такой же позиции с помощью \angstrom{} используется состояние
\begin{lstlisting}
    type 'a loc_state = {
        line       : int;
        line_start : int;
        custom     : 'a;
    }
\end{lstlisting}
и два парсера
\begin{lstlisting}
    val whitespace : unit t
    val position   : Lexing.position t
\end{lstlisting}
где \verb|position| возвращает текущую позицию, а \verb|whitespace| должен вызываться там где может встретиться перевод строки.

\newpage
\subsubsection{Производительность}

В таблице 3 --- время работы и потребляемая память \json-анализатора, использующего модифицированную версию \angstrom{}
с состоянием
\begin{lstlisting}
    module State = struct
        type t = unit loc_state
    end
\end{lstlisting}
относительно тех же значений для немодифицированной версии.

\begin{center}
    \begin{tabular}{|c|c|c|c|c|c|c|c|c|c|}
        \hline
                  & c1   & c2   & c3   & c4   & c5   & c6   & r    & r-c
        \\\hline
           Время  & 1.08 & 1.09 & 1.09 & 1.12 & 1.18 & 1.18 & 1.1  & 1.05
        \\ Память & 1.61 & 1.61 & 1.61 & 1.61 & 1.61 & 1.61 & 1.68 & 1.54
        \\\hline
    \end{tabular}
    \captionof{table}{Производительность модификации}
\end{center}
