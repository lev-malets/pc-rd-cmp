Компилятор \rescript{}, в основном, написан на \ocaml.
При этом его компиляция производится
компиляторами\footnote{Основной скрипт сборки: \url{https://github.com/rescript-lang/rescript-compiler/blob/9.1/scripts/ninja.js}}
из немного модифицированной версии \ocaml{}
4.06.1\footnote{Форк \ocaml: \url{https://github.com/rescript-lang/ocaml/tree/0090035986ddbcb91175195470d0e9d2edc632ab}}
(далее \ocamlbs).
Отсюда появляется основное ограничение для анализатора:
он должен компилироваться этой конкретной версией.

В данной работе для запусков тестов и измерений используются исполняемые файлы,
скомпилированные компилятором ocamlopt (компилятор в нативный код) из \ocamlbs.
Конфигурация для компиляции самого \ocamlbs{} взята из скриптов сборки \rescript
\footnote{Конфигурация для \ocamlbs: \url{https://github.com/rescript-lang/rescript-compiler/blob/9.1/scripts/buildocaml.js}}.

Для замеров времени работы и используемой памяти используется библиотека \verb=core_bench=.
Так как среди её зависимостей есть библиотека Thread (одна из стандартных библиотек), то из конфигурации \ocamlbs{}
был убран влаг \verb=-no-pthread=.
Это не должно влиять на производительность.
