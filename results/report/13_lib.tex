\subsection{Библиотеки}

Исходя их ограничения на версию, для рассмотрения были выбраны следующие библиотеки:
\angstrom \footnote{\angstrom : \url{https://github.com/inhabitedtype/angstrom}, дата доступа: 13 апреля 2021},
\opal     \footnote{\opal     : \url{https://github.com/pyrocat101/opal}       , дата доступа: 13 апреля 2021},
\mparser  \footnote{\mparser  : \url{https://github.com/murmour/mparser}       , дата доступа: 13 апреля 2021},
\bark     \footnote{\bark     : \url{https://github.com/justinlubin/bark}      , дата доступа: 13 апреля 2021}.
Это не полный список библиотек, подходящих по версии, однако прочие библиотеки на фоне выбранных ничем не выделяются.

\subsection{Сравнение}

С помощью каждой из библиотек был реализован анализатор \json{}.
Были измерены производительность и потребление памяти этих анализаторов при анализе нескольких \json-файлов.
В таблицах 1 и 2 приведены значения времени и памяти относительно разбора с помощью библиотеки \yojson{},
которая использует для синтаксического анализа \json{} сочетание утилиты
\textsc{ocamllex}\footnote{Аналог \textsc{UNIX} утилиты lex. Распространяется вместе с компилятором \ocaml} и рекурсивного спуска.

\noindent
Использованные файлы\footnote{\url{https://github.com/lev-malets/pc-rd-cmp}}:
\begin{lstlisting}
    c* = data/json/complex/*
    r  = data/json/others/repeat
    rc = data/json/others/repeat-compact
\end{lstlisting}

\noindent
\begin{center}
    \begin{tabular}{|c|c|c|c|c|c|c|c|c|c|}
        \hline
                        & c1  & c2  & c3  & c4  & c5  & c6  & r   & rc    \\\hline
        Angstrom        & 1   & 1   & 1   & 1.1 & 1.1 & 1.3 & 1.2 & 1.6   \\
        Opal            & 1.8 & 1.8 & 1.9 & 2.2 & 2.5 & 3.2 & 2.6 & 3.5   \\
        MParser         & 5.8 & 5.7 & 5.8 & 6   & 6.1 & 6.1 & 6.9 & 9.6   \\
        Bark            & 4.1 & 4   & 4.1 & 4.2 & 4.4 & 4.4 & 1.8 & 2.4   \\\hline
    \end{tabular}
    \captionof{table}{Время}
\end{center}

\noindent
\begin{center}
    \begin{tabular}{|c|c|c|c|c|c|c|c|c|c|}
        \hline
                        & c1   & c2   & c3   & c4   & c5   & c6   & r    & rc    \\\hline
        Angstrom        & 7.5  & 10.7 & 11   & 11.2 & 11.3 & 11.3 & 8.3  & 8.7   \\
        Opal            & 9    & 12.9 & 13.3 & 13.5 & 13.6 & 13.7 & 13.2 & 13.7  \\
        MParser         & 40.8 & 58.6 & 60.1 & 61.1 & 61.5 & 61.9 & 42.9 & 43.8  \\
        Bark            & 28   & 40.1 & 41.4 & 42   & 42.3 & 42.5 & 15.7 & 16.3  \\\hline
    \end{tabular}
    \captionof{table}{Память}
\end{center}

\subsection{Выбранная библиотека}

Исходя из результатов сравнения, для дальнейшей работы была выбрана библиотека \angstrom{}.
Для дальнейшей работы про \angstrom{} важно знать тип парсера:

\begin{lstlisting}[escapechar=!,language=ocaml]
    type 'a with_state = input -> 'a
    type 'a failure = (error -> 'a state) with_state
    type ('a, 'r) success = ('a -> 'r state) with_state

    type 'a t =
        { run : 'r. ('r failure -> ('a, 'r) success -> 'r state) with_state }
\end{lstlisting}
