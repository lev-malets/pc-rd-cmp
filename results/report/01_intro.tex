
При создании синтаксического анализатора может быть выбрана ручная реализация,
например, если не хватает возможностей генераторов синтаксических анализаторов.
Обычно в качестве метода реализации берётся \emph{рекурсивный спуск}.
Однако такие анализаторы отличаются объёмностью кода.
В пример анализаторов, реализованных вручную, можно привести
синтаксический анализатор в
\textsc{Clang}\footnote{\url{https://github.com/llvm/llvm-project/tree/llvmorg-12.0.0-rc5/clang/lib/Parse}}
или анализаторы в компиляторах
\textsc{TypeScript}\footnote{\url{https://github.com/microsoft/TypeScript/blob/v4.2.4/src/compiler/parser.ts}}
и
\textsc{Kotlin}\footnote{\url{https://github.com/JetBrains/kotlin/blob/v1.4.32/compiler/psi/src/org/jetbrains/kotlin/parsing/KotlinParsing.java}}.

Вместе с тем, существует родственный рекурсивному спуску подход, базирующийся на понятии \emph{парсер-комбинаторов}~\cite{monparsing}.
С его использованием анализатор описывается с помощью набора простых \emph{парсеров} и функций для их комбинации.
Подход парсер-комбинаторов полагается на использование функций высшего порядка,
и поэтому более распространен в экосистемах функциональных языков, например, \haskell{}, \scala{} и \ocaml{}.
Однако встречаются реализации и для классических объектно-ориентированных языков, такие как \textsc{Boost.Spirit} для \textsc{C++}.
Парсер-комбинаторы по сути есть высокоуровневый интерфейс для реализации синтаксических анализаторов,
поэтому получающиеся анализаторы будут компактнее, чем аналоги, реализованные с помощью рекурсивного спуска.
Эта высокоуровневость также позволяет использовать staging для повышения производительности анализатора\cite{parsley2020}.

В данной работе планируется переписать синтаксический анализатор
\rescript\footnote{\url{https://rescript-lang.org/}} с использованием подхода парсер-комбинаторов.
Планируется подтвердить, что, по сравнению с рекурсивным спуском,
парсер-комбинаторы позволяют получить более компактный, но и более медленный анализатор.
Предполагается численно оценить во сколько раз полученный синтаксический анализатор уменьшился и замедлился.
Анализатор \rescript{} был выбран, так как он, во-первых, реализован рекурсивным спуском,
и, во-вторых, реализован на функциональном языке программирования,
который позволяет естественным образом использовать подход парсер-комбинаторов.
