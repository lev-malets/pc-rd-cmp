\subsection{Модификация библиотеки}

\subsubsection{Состояние}

В тип парсера было добавлено состояние для возможности сохранения рабочей информации о ходе анализа.

\begin{lstlisting}[escapechar=!,language=Caml]
    type ('a, 'pstate) failure =
        input -> 'pstate -> string -> 'a state
    type ('a, 'r, 'pstate) success =
        input -> 'pstate -> 'a -> 'r state

    type ('a, 'pstate) t =
        { run : 'r. input -> 'pstate -> ('r, 'pstate) failure
            -> ('a, 'r, 'pstate) success -> 'r state
        }
\end{lstlisting}

Для взаимодействия с состоянием добавлен "стандартный" набор парсеров:

\begin{lstlisting}[escapechar=!,language=Caml]
    val get_state : state t
    val set_state : state -> unit t
    val map_state : (state -> state) -> unit t
\end{lstlisting}

\subsubsection{Позиция}

Для построения дерева разбора необходима возможность получения позиции в разбираемом тексте.

Были добавлены парсеры:

\begin{lstlisting}[escapechar=!,language=Caml]
    val whitespace : unit t
    val position   : Lexing.position t
\end{lstlisting}

\newpage
\subsubsection{Производительность}

После этих изменений производительность json-анализатора снизилась.

\begin{center}
    \begin{tabular}{|c|c|c|c|c|c|c|c|c|c|}
        \hline
                    & c1   & c2   & c3   & c4   & c5   & c6   & r    & r-c
        \\\hline
           Default  & 1.1  & 0.97 & 0.97 & 1.05 & 1.13 & 1.28 & 1.13 & 1.5
        \\ Modified & 1.26 & 1.15 & 1.23 & 1.28 & 1.43 & 1.62 & 1.34 & 1.66
        \\\hline
    \end{tabular}
    \captionof{table}{Производительность модификации}
\end{center}
